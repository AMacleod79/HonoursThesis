\chapter{Design}\label{ch:Design}

\section{Introduction}
\subsection{Problem Definition and outline}
Restate problem briefly & outline chapter structure
\subsection{Class Imbalance in healthcare datasets}
class imbalance and problems it can cause in healthcare datasets
\subsection{Proposed solutions}
what are the known solutions to the problem
\subsection{Experimental design outline}
overview of the experimental process (schema?)

\section{Dataset Choice and Preliminary Exploration}
\subsection{Datasets}
\subsubsection{How were the dataset chosen?}
what criteria were used to choose 7-10 datasets

\subsubsection{Brief Overview of the datasets}
brief description of datasets
\subsection{Preliminary Exploration of the chosen datasets}
\subsubsection{Defining features}
some quick visualisation of data 
\subsubsection{Class distribution}
show class distribution/imbalance for the chosen datasets


\section{Choice of algorithms used in evaluation}
\subsection{Quick overview of algorithms}
what algorithms will be used in our experiments
\subsection{Reasons for choosing those algorithms}
why are we choosing them

\section{Proposed avenues of exploration}
\subsection{Overview of typical methods to address class imbalance}
write a section on each method 
\subsection{Testing the Impact of the use of these methods}
\subsubsection{First test the algorithms on the dataset without any modifications}
Details of procedure which will be employed to get a baseline of performance for each dataset and algorithms before applying methods which will help correct the class imbalance

\subsubsection{Applying the methods to each datasets and algorithms}
Details of procedure which will be employed to adjust/correct class imbalance on the datasets


\section{Evaluating Performance of algorithms and correction methods}

\subsection{Metrics for evaluation}
Quick overview of each metric plus how helpful they are 
\subsubsection{Accuracy}
\subsubsection{Calibration}
\subsubsection{Discrimination}
\subsubsection{Negative Predictive Value}
\subsubsection{Precision}
\subsubsection{Recall}
\subsubsection{Specificity}

\subsection{Choice of most relevant metrics}
\subsubsection{which metrics will be most helpful in the healthcare context}
\subsubsection{create a combined metric for this evaluation}
this is maybe an idea, where a composite metric from the most appropriate metrics could be used?

\subsection{Comparing performance of algorithms for each dataset before and after applying the methods to compensate for class imbalance}
In this section detail how the performance of algorithms can be compared before any adjustments for class imbalances are made and after; details how each algorithms will be compared etc

\section{Technical Requirements}
\subsection{Language}
\subsection{Software}

\section{Discussion of results}
\subsection{Results obtained}
This section to detail how the results will be aggregated and discussed; how do we know we have achieved what we set out to do
\subsection{Driving factors for the results}
Discuss how different factors will be assessed
\subsection{Conclusions: most effective way to deal with class imbalance}
Details of how the most effective solutions will be decided from the results




\section{Conclusions}

This chapter has outlined the experimental design chosen for this project. 
What problem?
How do we evaluate?



